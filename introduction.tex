\chapter{Introduction}


Gravity measurements are one the instruments that geophysicists use to
investigate the subsurface of the Earth.
Gravity data can be acquired on the ground, airborne, shipborne or through
artificial satellites.
Ground, airborne, and shipborne data are routinely used in local or regional
studies, with applications ranging from archaeological studies
\citep[e.g., ][]{panisova2013} to mapping the depth of
sedimentary basins \citep[e.g., ][]{gordon2013}.
Satellite gravity data make continental and global scale studies possible
\citep[e.g., ][]{vandermeijde2013, vandermeijde2015, bouman2013,
braitenberg2015, reguzzoni2013}.
This is particularly important in regions where data acquisition by other means
is lacking or difficult to perform, such as South America, Africa, and
Antarctica.
Another advantage of satellite gravity measurements is the almost homogeneous
spacial coverage.
Satellite data also enable investigation of temporal variations of the Earth's
gravity field through the GRACE mission.
Applications using the time series data from GRACE include mapping ice-mass
variation in the Arctic \citep{chen2011} and Antarctic regions
\citep{ramillien2006}, quantifying deformation following large earthquakes
\citep{mikhailov2014}, and groundwater monitoring \citep{humphrey2016}.

Deriving geophysical Earth models from observed gravity data is an ill-posed
inverse problem.
Designing a method for solving these inverse problems presents many challenges.
The first challenge is to establish a functional mapping between the model
parameters and the data.
This is known as the forward problem and it must be solved in a stable and
accurate way for the inversion to succeed.
The second challenge is to choose and implement an optimization algorithm to
estimate the model parameters that best fit the observed data.
There are several well established optimization methods to choose from the
literature, for example, gradient-descent methods like the Gauss-Newton method
and Steepest Descent or stochastic  methods like the Genetic Algorithm.
Finally, there is the challenge of stabilizing the ill-posed inverse problem,
usually through regularization.
Different regularizing functions favor different model attributes, such as
smoothness or compactness.
Choosing an appropriate regularizing function is an indirect way to include
prior geological or geophysical information in the inverse problem.

When developing a new inversion methodology, one must implement in a software
application all three components of an inversion method (the forward problem,
optimization, and regularization).
Fortunately, these components can usually be uncoupled.
For example, the implementation of the forward problem does not depend on the
choice of optimization method or regularization.
Likewise, the software implementation of optimization requires only a knowledge
of a function to optimize (and possibly its derivatives), no matter what is the
forward problem or regularizing function.
Furthermore, changing the regularizing function used, in principle, does not
require changes to the implementations of the forward problem and the
optimization method.
Thus, the ideal software design is to have independent and reusable routines
for forward modeling, optimization, and regularization.
These three components can be combined in different ways to produce
new inversion software.

Here, we develop two software projects and apply their reusable components to
develop a 3D gravity inversion method in spherical coordinates.
Chapter~\ref{chap:tesseroids} describes the open-source software
\textit{Tesseroids}.
This C language program calculates the gravitational potential and its first
and second derivatives of a tesseroid (or spherical prism).
The software also improves upon existing algorithms for the forward modeling
calculations.
Chapter~\ref{chap:fatiando} describes the Python language library
\textit{Fatiando a Terra}.
The library contains a collection of functions and classes for inverse
problems, forward modeling, data and model visualization, and data processing.
The inverse problems tools implement optimization and regularization classes
that are uncoupled from specific forward problems.
These tools can be reused and combined in different ways to implement new
inversion methods.
\textit{Fatiando a Terra} also implements the tesseroid forward modeling
algorithm described in Chapter~\ref{chap:tesseroids}.
Finally, in Chapter~\ref{chap:moho} we build upon the foundation of
Chapters~\ref{chap:tesseroids} and \ref{chap:fatiando} to develop a novel
gravity inversion method.
The method estimates the depth of the crust-mantle interface (the Moho) from
observed gravity data using a spherical Earth approximation.
The software implementation of the inversion combines and extends the
optimization, regularization, and forward modeling available in
\textit{Fatiando a Terra}.
We apply our method to estimate the depth of the Moho for the South American
continent.
