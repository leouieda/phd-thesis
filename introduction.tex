\chapter{Introduction}


Gravity data as used in geophysics to investigate the subsurface.
Gravity data can be acquired on the ground, airborne, shipborne or in satellites.
The range of acquisition means that gravity can be used in a range of scales.
Ground and airborne are used in local or regional studies, from mining to
sedimentary basins.
Satellite data makes continental or global scale studies possible, particularly
in areas where ground airborne and ships are difficult to acquire
\citep[e.g., ][]{vandermeijde2013, vandermeijde2015, bouman2013,
braitenberg2015, reguzzoni2013}.
Satellite data also gives almost homogeneous coverage.
Satellite also allows investigation of temporal variations, particularly from
the GRACE mission.
Examples of applications of temporal variations include ice-mass variation in
the Arctic and Antarctic regions
\citep{chen2011, ramillien2006},
post-seismic deformations \citep{mikhailov2014},
and groundwater monitoring \citep{humphrey2016}.


Deriving geophysical Earth models from gravity data is an inverse problem.

Designing a method for solving these inverse problems presents many challenges.

The first stage in developing an inversion method is to establish the
functional mapping between the model parameters and the data, known as the
forward problem.

For the inversion to succeed, the forward problem must be solved in a stable
and accurate way.

