\chapter{Introduction}


Gravity data as used in geophysics to investigate the subsurface.

Gravity data can be acquired on the ground, airborne, shipborne or in satellites.

The range of acquisition means that gravity can be used in a range of scales.

Ground and airborne are used in local or regional studies (cite a few), from
mining (cite Dio etc) to sedimentary basins (cite some like barnes and val).

Satellite data makes continental or global scale studies possible, particularly
in areas where ground airborne and ships are difficult to acquire (cite south
america moho, africa and himalayas by carla, arabian peninsula, north sea
by ebbing).

Satellite data also gives almost homogeneous coverage.

Satellite also allows investigation of temporal variations, particularly from
the GRACE mission.

Examples of applications of temporal variations include ice-mass loss (cite
greenland, antartica, and arctic studies), post-seismic deformations (cite
something about the Chile earthquake), and recently groundwater monitoring
(cite some from the review).




Deriving density-contrasts and geometries from gravity data is an inverse
problem.

The first stage in developing an inversion method is to establish the
functional mapping between the model parameters and the data, known as forward
modeling.

Solving the forward problem in a stable and accurate way is a prerequisite for
inversion.

Different discretizations of the subsurface require different forward modeling
methods.

For local studies, a flat-Earth approximation and discretization into
right-rectangular prisms is the path often chosen (cite Li and Oldenburg,
Fernando, Cristiano, Barnes, etc).

Cite Nagy some way.

Then talk about spherical Earth and tesseroids. Also mention spherical
harmonics solution.
