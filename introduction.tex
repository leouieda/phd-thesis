\chapter{Introduction}


Gravity data is used in geophysics to investigate the subsurface.
Gravity data can be acquired on the ground, airborne, shipborne or in satellites.
The range of acquisition means that gravity can be used in a range of scales.
Ground and airborne are used in local or regional studies, from mining to
sedimentary basins.
Satellite data makes continental or global scale studies possible, particularly
in areas where ground airborne and ships are difficult to acquire
\citep[e.g., ][]{vandermeijde2013, vandermeijde2015, bouman2013,
braitenberg2015, reguzzoni2013}.
Satellite data also gives almost homogeneous coverage.
Satellite also allows investigation of temporal variations, particularly from
the GRACE mission.
Examples of applications of temporal variations include ice-mass variation in
the Arctic and Antarctic regions
\citep{chen2011, ramillien2006},
post-seismic deformations \citep{mikhailov2014},
and groundwater monitoring \citep{humphrey2016}.


Deriving geophysical Earth models from gravity data is an inverse problem.
Designing a method for solving these inverse problems presents many challenges.
The first challenge in developing an inversion method is to establish the
functional mapping between the model parameters and the data.
This is known as the forward problem.
The forward problem must be solved in a stable and accurate way for the
inversion to succeed.
The second challenge is to choose and implement an optimization algorithm to
estimate the model parameters that best fit the observed data.
There are several well established optimization methods to choose from the
literature, for example, the gradient descent Gauss-Newton method or the
stochastic Genetic Algorithm.
The optimization algorithm is problem agnostic, meaning that it does not depend
on the type of geophysical data or parametrization used.
Finally, there is the challenge of stabilizing the inverse problem.
This is often done through some form of regulatization.
Each regularizing function favors different attributes, such as smoothness or
compactness.
Choosing an appropriate regularizing function is an indirect way to include
prior geological or geophysical information.

When developing a new inversion methodology, one must implement in a software
application all three components of an inversion method (forward problem,
optimization, and regularization).
Luckily, the components can usually be uncoupled.
The forward problem does not depend on the optimization and regularization.
Likewise, the optimization algorithm requires only a knowledge of a function to
optimize (and possibly its derivatives), no matter what is the forward problem
or regularizing function.
Changing the regularizing function used in principle does not require changes
to the forward problem or the optimization method.
Thus, the ideal implementation would be to have independent and reusable
forward modeling, optimization, and regularization routines.

Here, we develop two software projects and apply their reusable components to
develop a 3D gravity inversion method in spherical coordinates.
Chapter~\ref{chap:tesseroids} describes the open-source software
\textit{Tesseroids}.
This C language program calculates the gravitational potential and its first
and second derivatives of a tesseroid (or spherical prism).
The software also improves upon existing algorithms for the forward modeling
calculations.
Chapter~\ref{chap:fatiando} describes the Python language library
\textit{Fatiando a Terra}.
The library contains a collection of functions and classes for forward
modeling, data and model visualization, data processing, and inversion.
The optimization and regularization components are uncoupled from specific
forward problems.
Thus, they can be reused and combined to implement new inversion methods.
\textit{Fatiando a Terra} also implements the tesseroid forward modeling
algorithm described in Chapter~\ref{chap:tesseroids}.
Finally, in Chapter~\ref{chap:moho} we build upon the foundation of
Chapters~\ref{chap:tesseroids} and \ref{chap:fatiando} to develop a novel
gravity inversion method.
The method estimates the depth of the crust-mantle interface (the Moho) in a
spherical approximation from observed gravity data.
The software implementation of the inversion combines and extends the
optimization, regularization, and tesseroid forward modeling available in
\textit{Fatiando a Terra}.
We apply our method to estimate the depth of the Moho for the South American
continent.
The estimated Moho depths agree with previous models and known large scale
tectonic features.
