\chapter{Conclusions}


We have developed two open-source software packages that implement the
components required for building an inversion method: forward modeling,
optimization, and regularization.
We used these components to develop a fast gravity inversion method to estimate
the depth of the crust-mantle interface (the Moho) in a spherical
approximation.
We then applied the proposed method to estimate the Moho depth for South
America.
Both software packages, the source code for the gravity inversion, and all
associated data are freely available on the Internet through various websites.

The \textit{Tesseroids} software is a collection of command-line
programs developed in the C programming language.
The programs calculate the gravitational potential and its first and second
derivatives of a tesseroid (spherical prism) model.
We implemented and improved upon an adaptive discretization algorithm to
guarantee the accuracy of the computations.
The adaptive discretization is controlled by a scalar called the distance-size
ratio ($D$).
Higher values of $D$ result in finner discretization and vice-versa.
Furthermore, we investigated the accuracy of the calculations as a function of
$D$.
Contrary to previous assumptions, our results showed that the first and second
derivatives require finner discretization than the gravitational potential to
achieve the same accuracy level.
The values of the distance-size ratio that yield a maximum error of 0.1\%
are $D = 1$ for the gravitational potential, $D = 1.5$ for the first
derivatives, and $D = 8$ for the second derivatives.
These values are included as defaults in version 1.2 of the software.
\textit{Tesseroids} has been used in published research by, for example,
\citet{braitenberg2011}, \citet{alvarez2012}, \citet{bouman2013},
\citet{bouman2013a}, \citet{mariani2013}, \citet{braitenberg2015}, and
\citet{fullea2015a}.

\textit{Fatiando a Terra} is a software library implemented in the Python
programming language.
The library contains functions and classes for data processing, visualization,
inversion, and forward modeling.
The inverse problems package of the library offers generic classes for
optimization and regularization.
These classes can be extended and combined with the existing forward modeling
functions to implement new inversion methods.
Using these tools, the amount of code required to implement a new method is
reduced, increasing the speed of the cycle of prototyping a new algorithm,
testing, and then refining it.
The project has been used in scholarly works such as
\citet{carlos2014}, \citet{hidalgo-gato2015a}, \citet{niccoli2015},
\citet{oliveirajr.2015}, and \citet{bassett2016}.
To date, the project has received contributions from nine developers in three
different countries.

We used the tesseroid forward modeling and inverse problem toolkit of
\textit{Fatiando a Terra} to implement a new non-linear gravity inversion
method to estimate the depth of the Moho.
The inversion method uses the Gauss-Newton formulation of Bott's method
proposed by \citet{silva2014}.
In this formulation, the Jacobian (sensitivity) matrix is approximated by a
linear diagonal matrix.
This eliminates the high computational cost of calculating the full dense
Jacobian matrix for every iteration of the Gauss-Newton method.
We stabilize the inverse problem through smoothness regularization.
The use of regularization required that we abandon the elimination of linear
systems of the traditional Bott's method.
However, we maintained computational efficiency by taking advantage of the fact
that all matrices involved are sparse.
Our benchmarks suggest that less than 0.1\% of the computation time required
for an inversion is spent on matrix operations and solving linear systems.
We also estimate the regularization parameter and two other hyper-parameters of
the inversion through cross-validation procedures.

The applications to synthetic data show the
efficiency of the proposed inversion method in retrieving smooth Moho depth
variations.
The cross-validation procedures are able to correctly estimate an optimal
regularization parameter, anomalous Moho density-contrast, and the reference
level (the depth of the Normal Earth Moho).
We applied the proposed method to estimate the Moho depth of the South American
continent.
The estimated Moho depths are in agreement with previous models and with the
major tectonic provinces of the continent.
Notable misfits between the observed and predicted gravity data are in the
Andes, northern Venezuela, and the Paraná, Solimões, and Amazon Basins.
These regions also present a high misfit with Moho depth estimates from
seismology.
Discrepancies between gravity-derived and seismological estimates of Moho depth
may indicate the presence of other density anomalies with the crust or upper
mantle.
For example, \citet{mariani2013} and \citet{nunn1988} explain the discrepancies
in the Paraná and Amazon Basins, respectively, as high density material in the
lower crust.

In conclusion,
we have successfully applied the algorithms and software foundation developed
here to solve a geophysical inverse problem.
This application demonstrates that the same approach can be used to aid the
development of new inversion methods in the future.
In addition, both software packages were successful in gaining third-party
users and have been applied to solve real world scientific problems.
\textit{Fatiando a Terra}, in particular, was able to gather other developers
to contribute source code to the software, enabling the continued growth of the
project in the future.
We emphasize that this is only possible because the project is open-source and
freely available.
