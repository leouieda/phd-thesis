\chapter{Conclusions}


We have developed two open-source software packages that implement the
components required for building an inversion method: forward modeling,
optimization, and regularization.
We used these components to develop a fast gravity inversion method to estimate
the depth of the crust-mantle interface (the Moho) in a spherical
approximation.
We then applied the proposed method to estimate the Moho depth for South
America.

The \textit{Tesseroids} software is a collection of command-line
programs developed in the C programming language.
The programs calculate the gravitational potential and its first and second
derivatives of a tesseroid (spherical prism) model.
We implemented and improved upon an adaptive discretization algorithm to
guarantee the accuracy of the computations.
The adaptive discretization is controlled by a scalar called the distance-size
ratio ($D$).
Higher values of $D$ result in finner discretization and vice-versa.
Furthermore, we investigated the accuracy of the calculations as a function of
$D$.
Contrary to previous assumptions, our results showed that the first and second
derivatives require finner discretization than the gravitational potential to
achieve the same accuracy level.
The values of the distance-size ratio that yield a maximum error of 0.1\%
are $D = 1$ for the gravitational potential, $D = 1.5$ for the first
derivatives, and $D = 8$ for the second derivatives.
These values are included as defaults in version 1.2 of the \textit{Tesseroids}
software.

\textit{Fatiando a Terra} is a software library implemented in the Python
programming language.
The library contains functions and classes for data processing, visualization,
inversion, and forward modeling.
The inverse problems package of the library offers generic classes for
optimization and regularization.
These classes can be extended and combined with the existing forward modeling
functions to implement new inversion methods.
Using these tools, the amount of code required to implement a new method is
reduced, increasing the speed of the cycle of prototyping a new algorithm,
testing, and then refining it.
The project has been used in scholarly works such as
\citet{carlos2014}, \citet{hidalgo-gato2015a}, \citet{niccoli2015},
\citet{oliveirajr.2015}, and \citet{bassett2016}.
To date, the project has received contributions from nine developers in three
different countries.
