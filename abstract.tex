\begin{abstract}

Apresenta-se, nesta tese, ...

\end{abstract}

\begin{foreignabstract}

We present methodological improvements to forward modeling and large-scale
inversion of satellite gravity data.
For this purpose, we developed two open-source software projects.
The first, is a C language suite of command-line programs called
\textit{Tesseroids}.
The programs calculate the gravitational potential, acceleration, and gradient
tensor of a spherical prism, or tesseroid.
\textit{Tesseroids} implements and extends an adaptive discretization algorithm
to automatically ensure the accuracy of the computations.
Our numerical experiments show that, to achieve the same level of accuracy, the
gravitational acceleration components require finner discretization than the
potential.
In turn, the gradient tensor requires finner discretization still than the
acceleration.
The second open-source project is \textit{Fatiando a Terra}, a Python language
library for inversion, forward modeling, data processing, and visualization.
The library allows the user to combine the forward modeling and inversion tools
to implement new inversion methods.
The gravity forward modeling tools include an implementation of the
\textit{Tesseroids} algorithm.
We combined the inversion and tesseroid forward modeling utilities of
\textit{Fatiando a Terra} to develop a new method for fast non-linear gravity
inversion.
The method estimates the depth of the crust-mantle interface (the Moho) based
on observed gravity data using a spherical Earth approximation.
We extended the computationally efficient Bott's method to include smoothness
regularization and use tesseroids instead right rectangular prisms.
The inversion is controlled by three hyper-parameters: the regularization
parameter, the Moho density-contrast, and the depth of the reference Moho.
We employ two cross-validation procedures to automatically estimate these
parameters.
Tests on synthetic data confirm the capability of the proposed method to
estimate smoothly varying Moho depths and the three hyper-parameters.
Finally, we applied our inversion method to produce a Moho depth model for
South America.
Our model fits the gravity data and seismological Moho depth estimates
in the oceanic areas and the central and eastern portions of the continent.
We observe large misfits in the Andes region, where Moho depth is largest.
In Amazon, Solimões, and Paraná Basins, the model fits the observed gravity
but disagrees with seismological estimates.
These discrepancies suggest the existence of density-anomalies in the crust or
upper mantle, as has been suggested in the literature.
\end{foreignabstract}
