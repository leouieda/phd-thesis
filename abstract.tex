\begin{abstract}

Apresentamos avanços metodológicos na área de modelagem direta e inversão
regional de dados de gravimetria por satélite.
Com esse fim, desenvolvemos dois projetos \DIFdelbegin \DIFdel{de software }\DIFdelend \DIFaddbegin \DIFadd{computacionais }\DIFaddend de código livre.
O primeiro é um conjunto de programas de linha de comando feitos na linguagem C
chamado \textit{Tesseroids}.
Os programas calculam o potencial, aceleração e tensor gradiente gravitacional
de um prisma esférico, ou tesseroide.
\textit{Tesseroids} implementa e aprimora um algoritmo de discretização
adaptativa para automaticamente garantir a acurácia das computações.
\DIFdelbegin \DIFdel{Nossos }\DIFdelend \DIFaddbegin \DIFadd{Os }\DIFaddend resultados com testes numéricos mostram que, para obter o mesmo nível de
acurácia, a aceleração gravitacional demanda uma discretização mais fina que o
potencial.
Por sua vez, o tensor gradiente gravitacional demanda discretização mais fina
ainda que a aceleração.
O segundo \DIFdelbegin \DIFdel{software }\DIFdelend \DIFaddbegin \DIFadd{projeto computacional }\DIFaddend é o \textit{Fatiando a Terra}, uma biblioteca
feita na linguagem Python para inversão, modelagem direta, processamento e
visualização de dados.
A biblioteca permite que o usuário combine as ferramentas de modelagem direta e
\DIFaddbegin \DIFadd{de }\DIFaddend inversão para implementar novos métodos de inversão.
As ferramentas de modelagem direta incluem uma implementação do algoritmo
utilizado no programa \textit{Tesseroids}.
\DIFdelbegin \DIFdel{Nós combinamos }\DIFdelend \DIFaddbegin \DIFadd{Combinamos }\DIFaddend os recursos de inversão e modelagem direta com tesseroides do
\textit{Fatiando a Terra} para desenvolver um método rápido para a inversão
não-linear de dados de gravidade.
O método estima a profundidade da interface crosta-manto (a Moho) baseado em
dados de gravidade utilizando uma aproximação esférica da Terra.
\DIFdelbegin \DIFdel{Nós adaptamos }\DIFdelend \DIFaddbegin \DIFadd{Adaptamos }\DIFaddend o método de Bott, que é computacionalmente eficiente, para
incluir regularização de suavidade  e utilizar tesseroides ao invés de prismas
retangulares retos.
A inversão é controlada por três hiper-parâmetros: o parâmetro de
regularização, o contraste de densidade \DIFdelbegin \DIFdel{na Moho }\DIFdelend \DIFaddbegin \DIFadd{entre a Terra real e o modelo de
referência (a Terra Normal) }\DIFaddend e a profundidade da Moho \DIFdelbegin \DIFdel{de
referência.
Nós aplicamos }\DIFdelend \DIFaddbegin \DIFadd{da Terra Normal.
Aplicamos }\DIFaddend dois tipos de validação cruzada para estimar esses parâmetros de
maneira automática.
Testes com dados sintéticos confirmam a capacidade do método proposto de
estimar os três hiper-parâmetros e o relevo suave da Moho.
\DIFdelbegin \DIFdel{Por fim, nós aplicamos nosso }\DIFdelend \DIFaddbegin \DIFadd{Finalmente, aplicamos o }\DIFaddend método de inversão \DIFaddbegin \DIFadd{desenvolvido }\DIFaddend para gerar um modelo de
profundidade da Moho para a América do Sul.
\DIFdelbegin \DIFdel{Nosso modelo }\DIFdelend \DIFaddbegin \DIFadd{O modelo de profundidade da Moho estimado }\DIFaddend ajusta os dados de gravidade
observados e \DIFaddbegin \DIFadd{as }\DIFaddend estimativas da profundidade da Moho provenientes da sismologia nas
regiões oceânicas e nas partes central e \DIFdelbegin \DIFdel{Leste }\DIFdelend \DIFaddbegin \DIFadd{leste }\DIFaddend do continente.
\DIFdelbegin \DIFdel{Nós observamos }\DIFdelend \DIFaddbegin \DIFadd{Observamos }\DIFaddend desajustes aos dados na região dos Andes, onde a profundidade da
Moho é a maior do continente.
Nas bacias do Amazonas, Solimões e Paraná, o modelo ajusta os dados de
gravidade mas não as estimativas da sismologia.
Essas discrepâncias indicam a presença de anomalias de densidade na crosta ou
manto superior, como sugerido anteriormente na literatura.
\end{abstract}

\begin{foreignabstract}

We present methodological improvements to forward modeling and regional
inversion of satellite gravity data.
For this purpose, we developed two open-source software projects.
The first is a C language suite of command-line programs called
\textit{Tesseroids}.
The programs calculate the gravitational potential, acceleration, and gradient
tensor of a spherical prism, or tesseroid.
\textit{Tesseroids} implements and extends an adaptive discretization algorithm
to automatically ensure the accuracy of the computations.
Our numerical experiments show that, to achieve the same level of accuracy, the
gravitational acceleration components require finner discretization than the
potential.
In turn, the gradient tensor requires finner discretization still than the
acceleration.
The second open-source project is \textit{Fatiando a Terra}, a Python language
library for inversion, forward modeling, data processing, and visualization.
The library allows the user to combine the forward modeling and inversion tools
to implement new inversion methods.
The gravity forward modeling tools include an implementation of the
algorithm used in the \textit{Tesseroids} software.
We combined the inversion and tesseroid forward modeling utilities of
\textit{Fatiando a Terra} to develop a new method for fast non-linear gravity
inversion.
The method estimates the depth of the crust-mantle interface (the Moho) based
on observed gravity data using a spherical Earth approximation.
We extended the computationally efficient Bott's method to include smoothness
regularization and use tesseroids instead right rectangular prisms.
The inversion is controlled by three hyper-parameters: the regularization
parameter, the \DIFdelbegin \DIFdel{Moho }\DIFdelend density-contrast \DIFaddbegin \DIFadd{between the real Earth and the reference model
(the Normal Earth)}\DIFaddend , and the depth of the \DIFdelbegin \DIFdel{reference Moho }\DIFdelend \DIFaddbegin \DIFadd{Moho of the Normal Earth}\DIFaddend .
We employ two cross-validation procedures to automatically estimate these
parameters.
Tests on synthetic data confirm the capability of the proposed method to
estimate smoothly varying Moho depths and the three hyper-parameters.
Finally, we applied \DIFdelbegin \DIFdel{our inversion method }\DIFdelend \DIFaddbegin \DIFadd{the inversion method developed }\DIFaddend to produce a Moho depth
model for South America.
\DIFdelbegin \DIFdel{Our }\DIFdelend \DIFaddbegin \DIFadd{The estimated Moho depth }\DIFaddend model fits the gravity data and seismological Moho
depth estimates in the oceanic areas and the central and eastern portions of
the continent.
We observe large misfits in the Andes region, where Moho depth is largest.
In Amazon, Solimões, and Paraná Basins, the model fits the observed gravity
but disagrees with seismological estimates.
These discrepancies suggest the existence of density-anomalies in the crust or
upper mantle, as has been suggested in the literature.
\end{foreignabstract}
